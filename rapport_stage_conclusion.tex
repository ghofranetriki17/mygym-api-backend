\newpage
\pagestyle{fancy}
\lhead{\textsc{Conclusion et perspectives}}
\renewcommand{\headrulewidth}{0.4pt}
\renewcommand{\footrulewidth}{0.4pt}
\begin{center}
{\LARGE{\bfseries \hspace {1 cm} Conclusion et perspectives }}\\[0.5em]
\end{center}

Ce projet a été mené dans le cadre de mon stage chez ExaDev à Sfax, avec pour objectif la conception et le développement de la plateforme ANAS’ GYM (API Laravel + dashboard React + app mobile React Native). L’enjeu était de proposer une solution multi-sites centralisant la gestion des salles, des cours collectifs, des équipements, des coachs, des workouts/programmes et du suivi des membres.

Plusieurs défis techniques et organisationnels ont été relevés : définir une architecture robuste (Laravel Sanctum, stockage public, pivots bookings/workouts), garantir la cohérence des données (capacités, historiques), et synchroniser les fronts (mobile et web) avec l’API en respectant les contraintes de performance et de sécurité.

La solution actuelle couvre les besoins identifiés, tout en restant ouverte à des évolutions. Pistes d’amélioration :
\begin{itemize}
    \item \textbf{Analytique avancée} : tableaux de bord plus riches (filtrage par période, KPI par branche/séance/coach), reporting exportable.
    \item \textbf{Optimisation continue} : monitoring (Grafana/Prometheus) des performances et erreurs, ajustements automatiques (caches, pagination, indexing BD).
    \item \textbf{Évolution des modèles} : affiner les règles métier (capacités, pénalités d’annulation, recommandations personnalisées) et, à terme, explorer des modèles prédictifs simples (fréquentation, no-show) pour améliorer la planification.
    \item \textbf{Tests renforcés} : automatisation (API + front), tests de charge sur les endpoints critiques (réservations, authentification), audits de sécurité réguliers.
    \item \textbf{Expérience utilisateur} : notifications plus intelligentes (rappels ciblés), améliorations UX mobile et dashboard, accessibilité.
\end{itemize}

Cette expérience a été riche techniquement (Laravel, React/React Native, CI/CD, stockage médias) et humainement (organisation agile, communication). Elle m’a permis de consolider mes compétences en ingénierie logicielle et en pilotage de projet. Je suis convaincu que ces acquis me préparent à relever de nouveaux défis et à faire évoluer la plateforme en phase avec les besoins des utilisateurs et du marché.
