\newpage
\begin{center}
    {\LARGE{\bfseries \hspace {2cm} Introduction Générale }}\\[1em]
\end{center}

Dans le contexte actuel où la pratique sportive et le suivi de la santé deviennent des priorités, les membres des salles de sport rencontrent souvent des difficultés pour gérer efficacement leurs entraînements, suivre leur progression physique et accéder aux informations sur les équipements et les coachs. Les programmes quotidiens sont rarement historisés, les données corporelles se perdent facilement et la réservation des cours collectifs reste complexe. La multiplicité des locaux et l’absence d’un système centralisé compliquent encore l’accès aux informations essentielles.

Le présent rapport synthétise le travail réalisé lors de mon **stage d’été de 2ᵉ année** au sein de l’entreprise ExaDev. L’objectif était de concevoir et développer la plateforme **ANAS’ GYM** (code interne : MyGym) comprenant une application mobile pour les membres (React Native / Expo) et un tableau de bord administrateur (React) adossés à une API Laravel (Sanctum). Cette plateforme vise à centraliser la gestion des entraînements, la réservation des cours collectifs, le suivi des progrès physiques, la consultation des coachs et la gestion des équipements disponibles dans chaque local.

Le développement s’est articulé autour de trois volets :
- Une interface mobile intuitive pour les membres (réservation, suivi, notifications).
- Un dashboard administrateur pour la gestion centralisée des locaux, coachs, machines et séances.
- Un système d’historisation et de stockage des données (workouts, mesures corporelles, paramètres publics des salles).

Ce stage a mobilisé des compétences en développement mobile (React Native / Expo), web (React), et backend (Laravel + Sanctum). Le rapport retrace l’étude préalable et l’analyse des besoins, puis la conception (modélisation, diagrammes de classes et de use cases), jusqu’à la réalisation technique et à la mise en œuvre de la solution ANAS’ GYM.

La structure retenue est la suivante :
1. Présentation de l’entreprise d’accueil et contexte du projet.
2. Étude préalable : besoins, contraintes et pistes de solutions.
3. Étude conceptuelle : spécification fonctionnelle, modélisation (diagrammes de classes et use cases), architecture technique, choix technologiques.
4. Réalisation : implémentation de l’application mobile et du dashboard administrateur, principales fonctionnalités et intégration API Laravel, suivies d’une conclusion et des perspectives d’évolution.
